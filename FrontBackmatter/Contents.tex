%*******************************************************
% Table of Contents
%*******************************************************
\pagestyle{scrheadings}
%\phantomsection
\pdfbookmark[1]{\contentsname}{tableofcontents}
\setcounter{tocdepth}{2} % <-- 2 includes up to subsections in the ToC
\setcounter{secnumdepth}{3} % <-- 3 numbers up to subsubsections
\manualmark
\markboth{\spacedlowsmallcaps{\contentsname}}{\spacedlowsmallcaps{\contentsname}}
\tableofcontents
\automark[section]{chapter}
\renewcommand{\chaptermark}[1]{\markboth{\spacedlowsmallcaps{#1}}{\spacedlowsmallcaps{#1}}}
\renewcommand{\sectionmark}[1]{\markright{\textsc{\thesection}\enspace\spacedlowsmallcaps{#1}}}
%*******************************************************
% List of Figures and of the Tables
%*******************************************************
\clearpage
% \pagestyle{empty} % Uncomment this line if your lists should not have any headlines with section name and page number
\begingroup
    \let\clearpage\relax
    \let\cleardoublepage\relax
    %*******************************************************
    % List of Figures
    %*******************************************************
    %\phantomsection
    %\addcontentsline{toc}{chapter}{\listfigurename}
    % \pdfbookmark[1]{\listfigurename}{lof}
    % \listoffigures

    % \vspace{8ex}

    %*******************************************************
    % List of Tables
    %*******************************************************
    %\phantomsection
    %\addcontentsline{toc}{chapter}{\listtablename}
    % \pdfbookmark[1]{\listtablename}{lot}
    % \listoftables

    % \vspace{8ex}
    % \newpage

    %*******************************************************
    % List of Listings
    %*******************************************************
    %\phantomsection
    %\addcontentsline{toc}{chapter}{\lstlistlistingname}
    % \pdfbookmark[1]{\lstlistlistingname}{lol}
    % \lstlistoflistings

    % \vspace{8ex}

    % %*******************************************************
    % % Acronyms
    % %*******************************************************
    % %\phantomsection
    % \pdfbookmark[1]{Acronyms}{acronyms}
    % \markboth{\spacedlowsmallcaps{Acronyms}}{\spacedlowsmallcaps{Acronyms}}
    \chapter*{Acronyms}
    \begin{tabular}{ll}
      BOLD & Blood oxygenated level dependent \\
      EEG & Electroencephalography \\
      fMRI & Functional magnetic resonance imaging \\
        ICA & Independent component analysis \\
      MEG & Magnetoencephalography \\
        ML & Maximum likelihood \\
      MMSE & Minimum mean squared error \\
      MNI template & Montreal Neuroscience Institute template \\
      MVICA & MultiViewICA \\
      PCA & Principal component analysis \\
      RAM & Random access memory \\
      ShICA & Shared ICA \\
      SRM & Shared response model \\
      SVD & Singular value decomposition \\
    \end{tabular}

    \chapter*{Notations}
    We write vectors in bold letter $\vb$ and scalars in lower case $a$. Upper case letters $M$ are used to denote
    matrices. We denote $|W|$ the absolute value of the determinant of $W$. $\xb \sim \Ncal(\mub, \Sigma)$ means that $\xb \in \mathbb{R}^k$ follows a multivariate normal distribution of mean $\mub \in \mathbb{R}^k$ and
    covariance $\Sigma \in \mathbb{R}^{k \times k}$. When $\xb \sim \Ncal(\mub,
    \Sigma)$, its density is given by $\xb \rightarrow \Ncal(\xb; \mub, \Sigma)$. The $j, j$ entry of a diagonal matrix $\Sigma_i$ is denoted $\Sigma_{ij}$, the $j$ entry of $\yb_i$ is denoted $y_{ij}$. $\delta$ is the Kronecker delta.
    We use the usual scalar product for matrices $\dotp{ A }{ B} =
    \tr(A^{\top} B)$ and the associated norm is denoted $\|A\| = \sqrt{\dotp{
        A}{ A }}$. Vectors can be seen as tall matrices and
    therefore the scalar product and the norm are the same as for matrices.
    The gradient of a real function $f(\xb) \in \RR$ is denoted $\partialfrac{\xb}{f(\xb)}$
    and is seen as a column vector. The Jacobian of a vector valued
    function $\fb(\xb)$ is denoted $\partialfrac{\xb}{\fb(\xb)}$ and is a matrix
    such that the line $j$ is given by $\partialfrac{\xb}{\fb(x_j)}^{\top}$ where $x_j$
    is the $j$-th coordinate of $\xb$.
\endgroup
