%*******************************************************
% Abstract
%*******************************************************
%\renewcommand{\abstractname}{Abstract}
\pdfbookmark[1]{Abstract}{Abstract}
% \addcontentsline{toc}{chapter}{\tocEntry{Abstract}}
\begingroup
\let\clearpage\relax
\let\cleardoublepage\relax
\let\cleardoublepage\relax

\chapter*{Abstract}
In neuroscience, a rising trend is to use naturalistic paradigms, such as movie
watching or audio track listening, that are
unconstrained from behavioral manipulations and thus, more ecological to
real-world conditions.
However, the stimulations are challenging to model in this context and therefore,
the statistical analysis of the data using supervised regression-based
approaches is difficult.
This has motivated the use of unsupervised learning methods that do not make
assumptions about what triggers brain activations in the presented stimuli.

In this thesis, we first consider the case of the shared response model (SRM), where
subjects are assumed to share a common response. While this algorithm is useful
to perform dimension reduction, it is particularly costly on functional magnetic
resonance imaging (fMRI) data where the
dimension can be very large (on the order of 100~000). We considerably speed up the
algorithm and reduce its memory usage.

However SRM relies on assumptions that are not biologically plausible. In
contrast, independent component analysis (ICA) is more realistic but not suited
to multi-subject datasets. In this thesis, we present a well principled method
called MultiViewICA that extends ICA to datasets containing multiple subjects: MultiViewICA.
But while MultiViewICA is a maximum likelihood estimator but with a closed form likelihood
that can be efficiently optimized, it assumes the same amount of noise for all
subjects.
We therefore introduce ShICA, a generalization of MultiViewICA that comes with a
more general noise model. In contrast to almost all ICA-based model, ShICA can
separate Gaussian and non-Gaussian sources and comes with a minimum mean square
error estimate of the common sources that weights each subject according to its
estimated noise level.
In practice, MultiViewICA and ShICA yield on magnetoencephalography and
functional magnetic resonance imaging a more reliable estimate
of the shared response than competitors.

Lastly, we use independent component analysis as a basis to perform data
augmentation.  More precisely, we introduce CondICA, a data augmentation method
that leverages the large amount of unlabeled fMRI data to build a generative
model for labeled data using only a few labeled samples. CondICA yields an
increase in decoding accuracy on eight large fMRI datasets.

\pagebreak 

\begin{otherlanguage}{french}
\pdfbookmark[1]{Resume}{Resume}
\chapter*{Résumé}
En neuroscience, les stimulis naturels comme le visionnage d'un film ou
l'écoute de piste audio sont de plus en plus utilisés car ils sont exempts de
toute manipulation comportementale et de ce fait plus proches du monde réel. 
Toutefois, ces stimulis sont compliqués à modéliser et à analyser car l'utilisation de
méthodes supervisées fondées sur des régressions est difficile.
C'est ce qui motive l'utilisation de methodes non-supervisées qui ne font pas
d'hypothèses sur ce qui déclenche les activations neuronales.

Dans cette thèse, nous considérons d'abord le cas du modèle de réponse partagée (SRM), dans lequel
les sujets sont supposés partager une réponse commune. Cet algorithme est utile pour
réduire la dimension des données, mais il est coûteux pour les données
d'imagerie fonctionnelle (IRMf) où la dimmension peut être immense (de l'ordre
de 100~000).
Nous présentons dans cette thèse une version bien plus rapide et beaucoup plus
économe en mémoire.

Mais le SRM fait des hypothèses irréalistes sur les données d'imageries. Des
hypothèses plus réaliste sont utilisées dans l'analyse en composante
indépendente (ICA) mais cette méthode est difficile à généraliser aux jeux de
données qui contiennent plusieurs sujets. Nous proposons alors une extension de
l'ICA  appelée MultiViewICA, fondée sur la principe de maximum de vraisemblance
et qui convient à des jeux de données multi-sujets.
MultiViewICA a une vraisemblance en forme fermée qui peut être maximisée
efficacement. Toutefois, cette méthode suppose la même quantité de bruit pour tous les sujets.
Nous présentons donc ShICA, une généralisation de MultiViewICA qui s'accompagne d'un modèle de bruit plus général.
Contrairement à presque tous les modèles fondés sur l'ICA, ShICA peut
séparer des sources gaussiennes et non gaussiennes et propose une estimation
optimale des sources communes (au sens des moindres carrés), qui pondère chaque
sujet en fonction de son niveau de bruit estimé.
En pratique, MultiViewICA et ShICA permettent d'obtenir, en magnétoencéphalographie et en IRMf, une estimation plus fiable de la réponse commune que leurs concurrents.

Enfin, nous utilisons l'ICA comme base pour
faire de l'augmentation de données. Plus précisément, nous présentons CondICA,
une méthode d'augmentation de données qui exploite la grande quantité de données
d'IRMf non étiquetées pour construire un modèle génératif un utilisant seulement
un petit nombre de données étiquetées.
CondICA permet d'augmenter la précision du décodage sur huit grands jeux de données d'IRMf.
\end{otherlanguage}

\endgroup

\vfill
