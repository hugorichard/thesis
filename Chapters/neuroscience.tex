\section{Functional magnetic resonance imaging (fMRI)}
Task fMRI:
As a non-invasive brain imaging technique, task fMRI records brain
activity while participants are
performing specific cognitive tasks.

\subsection{Principle}
\begin{itemize}
\item 1990 Ogawa observed BOLD effect
\item 1992 First images in Human
  \item Blood need energy (sugar oxygen) quickly, Blood releases oxygen that is
    not consumed (hemodynamic repsonse) Deoxyhemoglobin produces distortion of
    the magnetic field. Peaks after 5s.
  \item Block designs
  \item Resting state
    \item Naturalistic stimuli
\end{itemize}
\subsection{Pre-processing}
\begin{itemize}
\item Slice time correction
\item Spatial re-alignment
\item Coregistration to the T1 image
\item Affine transformation of the functional volumes to a template brain (MNI)
\end{itemize}
\subsection{A classical supervised experimental setting to map brain functions}
\subsubsection{Block design}
Describe block design setting and GLM model
\subsubsection{GLM model}
Univariate statistical methods, such as general linear models (GLMs)
\cite{friston1995analysis} have been successfully applied to
identifying the brain regions involved in specific tasks.
% 
Describe seed based resting state as well as atlas based resting state.

\subsubsection{Decoding}
However such methods do not capture well correlations and interactions between brain-wide measurements.
By contrast, classifiers trained 
to \emph{decode} brain maps, i.e to discriminate between specific stimulus or task types~\cite{shirer_decoding_2012,varoquaux_how_2014,loula_decoding_2018}, take these correlations into account. 
% 
The same framework is also popular for individual imaging-based diagnosis.

% 
\subsection{An example of unsupervised design: analysis of resting state data}

\section{Magneto electro encephalography (MEG)}
\subsection{Principle}
History
\begin{itemize}
  \item EEG: 1929 Hans Berger developed electroencephalograpgy, the grapgic
    representation of the difference in voltage between towo different cerbral
    locations plotted over time. (Electrical potential difference)
    \item 1972: David Cohen invented magneto encephalography. This measurement
      takes advantage of the direct connection between electrical activity and
      magnetism. (Magnetic field generated by neural activity)

      \item When a group of neuron (50 000~100 000) is simulatenously excited, it generates a current
      of large enough magnitude to be captured by EEG/MEG
    \item Artifacts: signals not generated by the brain. Cleaning the data two types: physiological ones (eye movement (blinking), muscal artefact
      (squizing teath)), non physiological ones (cell phone) 
     \item Forward problem: measurement of the current from the sources (rather
       easy). Model electrical condution of human head.  Difficult problem is to
       find the sources from the MEG/EEG (inverse problem). More possible
       solutions for sources than number of electrodes.
       Making reasonnable guess on sources. No unique solution.

     \item Signal neads to pass protective layers (becomes weaker). We have
       active electrodes with amplifiers inside or passive that is amplified
       only later (EEG)
     \item MEG: SQUID (measuring very weak magnetic fields) and applify a few centimers away from the sensors.
      
\end{itemize}
\subsection{Preprocessing}
Predefined distance from the screen, and for a predifined task. Blank screen in
between stimulus. Phase lock activity  is generated. Average across trials = ERP
(event related potential). Positive deflection P100.
Good temporal resolution bad spatial resolution.
Stimulation can be shown continuously. We then present a stimulkus flickering at
a specific frequency.
\subsection{Source localization: forward and inverse models}
Talk about how source localization is done in MEG. Insist on the fact that it is
rather slow so that in the next section we can make the link with ICA
