\section{Independent component analysis}
\subsection{Non Gaussian ICA}
\subsubsection{Identifiability}
\subsubsection{Robustness to density missmatch}
\subsubsection{Stability (extended infomax)}
\subsection{non-stationary ICA}
\subsubsection{Identifiability}
\subsubsection{Joint diagonalization objective}
\subsection{Dimension reduction}
So far, we have assumed that the dimensionality of the data $v$ and the number
of components $p$ is the same. 
In practice, however, we might want to estimate fewer components than there are observations per view; the original dimensionality of the data %(number of voxels, sensors) 
might in practice not be computationally tractable.

% The problem of how to perform subject-wise dimensionality reduction in group studies 
% % of data for each of the individuals while still considering them jointly and preserving the signal shared across them
% is an interesting one \emph{per se}, and out of the main scope of this work. For our purposes, it can be considered as a preprocessing step for which well-known various solutions can be applied. % step prior to the application of our method,. 
% We discuss this further in section~\ref{sec:rel_work} and in appendix~\ref{sec:app_rel_work}.

A classical approach to perform dimension reduction is principal component
analysis (PCA) which we now present.
Our data is represented as a random vector $\xb \in \bbR^v$ that is assumed
centered ($\EE[\xb] = 0$). Take $C = \Cov(\xb) \in \mathbb{R}^{v \times v}$ the
covariance of $\xb$ and compute an eigenvalue decomposition $VD^2V {\top} = C$
such that the coefficients of the diagonal matrix $D^2 \in \mathbb{R}^{v
  \times v}$ are ordered in decreasing order and $V$ is an orthogonal matrix.
Then $V = [\vb_1 \dots \vb_m]$ gives the projection vectors such that the $p$ first
principal components are then given by $[\vb_1 \dots \vb_p]^{\top} \xb$. Note
that our formulation of PCA does not include whitening of the signals.



% For a zero-mean data matrix $X$ of size $p\times n$ with $p \leq n$, we denote $X= UD V^{\top}$ the singular value decomposition of $X$ where $U \in \bbR^{p\times p}$, $V \in \bbR^{n \times p}$ are orthogonal and $D$ the diagonal matrix of singular values ordered in decreasing order.
% The PCA of $X$ with $k$ components is $Y\in\bbR^{k\times n}$ containing the
% first $k$ rows of $DV^{\top}$, and it does not hold in general that
% $YY^{\top}=I_k$: in this thesis, what we cann PCA does not include whitening of the signals.
\section{Analysis of MultiView data}
Many methods for data-driven multivariate analysis of neuroimaging group studies have been proposed. We summarize the characteristics of some of the most commonly used ones. A more thorough description of these methods can be found in appendix~\ref{sec:app_rel_work}.
\subsection{Group independent component analysis}
\subsubsection{Likelihood based method}
One can consider the more general model $\xb_i = A_i\sbb^i + \nb_i$, where the noise covariance can be learned from the data~\cite{guo2008unified}.
% 
Having the simpler model~\eqref{eq:mvica:model} leads to a closed-form likelihood, that can then be optimized by more efficient means than the EM algorithm.
In model~\eqref{eq:mvica:model}, the noise can be interpreted as individual variability rather than sensor noise. %It offers a way to capture more structured noise as is often the case in brain signals.
% It offers a way to capture more structured noise, which is often present in neuroimaging recordings~\cite{engemann2015automated}.
In appendix~\ref{app:complex_cov}, we generate data following model $\xb_i = A_i\sbb^i + \nb_i$ and report the reconstruction error. The difference in performance between algorithms is small. 
\subsubsection{CanICA}
\cite{varoquaux2009canica}

\subsubsection{ConcatICA}
When datasets are high-dimensional, a three steps procedure is often used: first dimensionality reduction is performed on data of each subject  separately; then the reduced data are merged into a common representation; finally, an ICA algorithm is applied for shared components extraction. The merging of the reduced data is often done by PCA \cite{calhoun2001method} or multi set CCA \cite{varoquaux2009canica}.
%Note that even with large datasets, it can still be computationally feasible to do group level reduction in one step (see \cite{chen2015reduced} or \cite{smith2014group}).
This is a popular method for fMRI~\cite{calhoun2009review} and EEG~\cite{eichele2011eegift} group studies.
These methods directly recover only group level, shared components; when individual components are needed, additional steps are required (back-projection \cite{calhoun2001method} or dual-regression \cite{beckmann2009group}).
%
In contrast, MultiView ICA finds individual and shared independent components in a single step.
%
%
Finally, in contrast to the methods described above, our method maximizes a likelihood, which brings statistical guarantees like consistency or asymptotic efficiency.
\subsection{Independent vector analysis}
\subsubsection{IVA-L}
\subsubsection{IVA-G}

\subsection{Multiset CCA}

\subsection{Hyperalignment}

\subsection{The shared response model (SRM)}
The shared response model~\cite{chen2015reduced} is a multi-view latent factor
model. The data $\xb_1 \dots \xb_m$ are modeled as random vectors following the model:
\begin{align}
 &\xb_i = A_i \sbb + \nb_i \\
  &A_i^{\top}A_i = I_p
  \label{eq:model:srm}
\end{align}
where $\xb_i \in \RR^v$ is the data of view $i$, $A_i \in \RR^{p, v}$ is the
mixing matrix of view $i$, $\nb_i$ is the noise of view $i$ and $\sbb$ are the
shared components referred to as the \emph{shared response} in fMRI applications.
The mixing matrices
$A_i$ are assumed to be orthogonal so that $A_i^{\top}A_i = I_p$. However in
general the matrix $A_i A_i^{\top}$ is different from identity. The noise
$\nb^i$ is assumed to be Gaussian with covariance $\Sigma_i$ and independent
across views. We assume the number of features $v$ to be much larger than the
number of components $p$: $v >> p$.

The conceptual figure~\ref{fig:srm:conceptual_figure} illustrates an 
application of shared response modeling to fMRI data. The mixing
matrices are spatial topographies specific to each subjects while the shared
components give the common timecourses.

In~\cite{chen2015reduced, anderson2016enabling}, two versions of the shared response model are
introduced which we now present.
\subsubsection{Deterministic shared response model}
The deterministic shared response model sees both $A_i$ and $\sbb$ as parameters to be
estimated and $\forall i, \Sigma_i=\sigma^2 I_v$ where $\sigma$ is an hyper-parameter.
The model is optimized by maximizing the log-likelihood.
The likelihood is given by: $p(\xb) = \prod_i \Ncal(\xb_i; A_i \sbb, \sigma^2 I)$ and
therefore the negative log-likelihood is given up to a constant independent of
$A_i$ and $\sbb$ by:
\begin{align}
  \loss = \sum_i \|A_i \sbb - \xb_i \|^2 = \| \sbb \|^2 -2 \langle A_i \sbb, \xb_i \rangle + \| \xb_i \|^2
  \label{eq:detsrmloss}
\end{align}
The negative log-likelihood $\loss$ is optimized by performing alternate minimization on $(A_1 \dots A_m)$
and $\sbb$. Note that the hyper-parameter $\sigma$ does not have an influence on
the loss and can therefore be safely ignored.

The gradient with respect to $\sbb$ is given by $ \sum_i A_i^{\top}(A_i \sbb - \xb_i)$
yielding the closed form updates:
\begin{equation}
  \sbb \leftarrow  \frac1m \sum_i (A_i^{\top} \xb_i)
  \label{eq:srm:supdate}
\end{equation}

From \eqref{eq:detsrmloss}, minimizing $\loss$ with respect to $A_i$ is
equivalent to maximize $\langle A_i, \xb_i \sbb^{\top} \rangle$ and therefore we
have:
\begin{equation}
  A_i \leftarrow  \Pcal(\xb_i \sbb^T)
  \label{eq:detsrm:Aiupdate}
\end{equation}
where $\Pcal$ is the projection on the Stiefel manifold: $\Pcal(M) = M
(M^{\top}M)^{-\frac12}$. In practice $\Pcal(M)$ is computed by performing an SVD
of $M$, $M = U_M D_M V_M^{\top} $ so that $\Pcal(M) = U_M V_M^{\top}$.

\subsubsection{Probabilistic SRM}
In Probabilistic SRM , $\Sigma_i=\sigma_i^2 I_v$ and the shared
components are assumed to be Gaussian $\sbb \sim \Ncal(0, \Sigma_s)$.
As will be seen in the next section, the model is identifiable only if
$\Sigma_s$ is diagonal with different diagonal values so we now assume
$\Sigma_s$ follows these two requirements.

The model is optimized via the expectation maximization algorithm.
Denoting $\VV[\sbb | \xb] = (\sum_i \frac1{\sigma_i^2} I +
\Sigma_s^{-1})^{-1}$ and $\EE[\sbb | \xb] = \VV[\sbb | \xb] \sum_i \frac1{\sigma_i^2}
A_i^{\top}\xb_i$, we have
\begin{align}
  p(\xb, \sbb) &= \prod_i \frac{\exp(-\frac{\|\xb_i - A_i \sbb \|^2}{2 \sigma_i^2})}{(2 \pi \sigma_i^{2v})^{\frac12}} \frac{\exp(-\frac12 \langle \sbb , \Sigma_s^{-1} \sbb \rangle )}{(2 \pi | \Sigma_s|)^{\frac12}} \\
               &= c_1 \exp(-\frac12 \left( \sum_i \frac1{\sigma_i^2}\|\xb_i\|^2 - 2  \langle \sum_i \frac1{\sigma_i^2} A_i^{\top}\xb_i, \sbb \rangle + \sum_i \frac1{\sigma_i^2} \| \sbb \|^2 + \langle \sbb, \Sigma_s^{-1} \sbb \rangle  \right)) \\
               &= c_2 \exp(-\frac12 \left( \langle  \sbb - \EE[\sbb | \xb], \VV[\sbb | \xb]^{-1} ( \sbb - \EE[\sbb | \xb])  \rangle \right)) \\
\end{align}
where $c_1, c_2$ are independent of $\sbb$.
Therefore $\sbb| \xb \sim \Ncal(\EE[\sbb | \xb], \VV[\sbb, \xb])$

The negative completed log-likelihood is given by
\begin{align}
	\loss = \sum_i \frac12 v \log(\sigma_i^2) + \frac1{2 \sigma_i^2} \| \xb_i - A_i \sbb \|^2
\end{align}
updates are therefore given by:
\begin{align}
&\sigma_i^2 \leftarrow \frac1{v} (\| \xb_i - A_i \EE[\sbb|\xb]\|^2 + \| \diag(\VV[\sbb | \xb]) \|^2) \\
  &A_i \leftarrow \Pcal(\xb_i \EE[\sbb|\xb]^{\top}) \\
  & \Sigma_s \leftarrow \VV[\sbb | \xb] + \EE[\sbb | \xb] \EE[\sbb | \xb]^{\top}
    \label{eq:srm:Aiupdate}
\end{align}

The complexity of Probabilistic SRM or Deterministic SRM is in $\bigO(\mathrm{n_{iter}} mpvn)$ where
$n$ is the number of samples and $\mathrm{n_{iter}}$ the number of iterations
performed until convergence is reached while the storage requirements are in $\bigO(mvn)$.

% \section{Related Work}
% \label{sec:rel_work}
% %

% %
% %

% %
% %
% %

% \textbf{Structured mixing matrices} One strength of our model is that we only assume that the mixing matrices are invertible and still enjoy identifiability whereas some other approaches impose additional constraints. For instance tensorial methods~\cite{beckmann2005tensorial} assume that the mixing matrices are the same up to diagonal scaling.
% %
% Other methods impose a common mixing matrix~\cite{cong2013validating, grin2010independent, calhoun2001fmri, Monti18UAI}. Like PCA, the Shared Response Model~\cite{chen2015reduced} (SRM) assumes orthogonality of the mixing matrices. While the model defines a simple likelihood and provides an efficient way to reduce dimension, the SRM model is not identifiable as shown in appendix~\ref{sec:app_identifiability}, and the orthogonal constraint may not be plausible.
% %

% \textbf{Matching components a posteriori} A different path to multi-subject ICA is to extract independent components with individual ICA in each subject and align them. We propose a simple baseline approach to do so called \emph{PermICA}.
% Inspired by the heuristic of the hyperalignment method~\cite{haxby2011common} we choose a reference subject and first match the components of all other subjects to the components of the reference subject. The process is then repeated multiple times, using the average of previously aligned components as a reference. Finally, group components are given by the average of all aligned components. We use the Hungarian algorithm to align pairs of mixing matrices~\cite{tichavsky2004optimal}.
% %
% Alternative approaches involving clustering have also been developed~\cite{esposito2005independent,bigdely2013measure}.

% \textbf{Deep Learning} Deep Learning methods, such as convolutional auto-encoders (CAE), can also be used to find the subject specific unmixing~\cite{chen2016convolutional}. While these nonlinear extensions of the aforementioned methods are interesting, these models are hard to train and interpret. In the experiments on fMRI data in appendix~\ref{appendix_reproduce}, we obtain better accuracy with MultiView ICA than that of CAE reported in~\cite{chen2016convolutional}.

% \textbf{Correlated component analysis} Other methods can be used to recover the shared neural responses such as the correlated component approach of Dmochowski~\cite{dmochowski2012correlated}. We benchmark our method against its probabilistic version~\cite{kamronn2015multiview} called BCorrCA in Figure~\ref{fig:meg}. Our method yields much better results. 

% \textbf{Autocorrelation} Another way to perform ICA is to leverage spectral diversity of the components rather than non-Gaussianity.
% %
% These methods are popular alternative to non-Gaussian ICA in the single-subject setting~\cite{tong1991indeterminacy, belouchrani1997blind, pham1997blind} and they output significantly different components than non-Gaussian ICA~\cite{delorme2012independent}.
% %
% Extensions to multiview problems have been proposed~\cite{lukic2002ica, congedo2010group}.
% \vspace{-5pt}